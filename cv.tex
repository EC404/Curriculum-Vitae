\documentclass[letterpaper,11pt]{article}

\usepackage{xeCJK}
\usepackage{latexsym}
\usepackage[empty]{fullpage}
\usepackage{titlesec}
\usepackage{marvosym}
\usepackage[usenames,dvipsnames]{color}
\usepackage{verbatim}
\usepackage{enumitem}
\usepackage[hidelinks]{hyperref}
\usepackage{fancyhdr}
\usepackage[english]{babel}
\usepackage{tabularx}
\usepackage{fontawesome5}

\pagestyle{fancy}
\fancyhf{} % clear all header and footer fields
\fancyfoot{}
\renewcommand{\headrulewidth}{0pt}
\renewcommand{\footrulewidth}{0pt}

% Adjust margins
\addtolength{\oddsidemargin}{-0.5in}
\addtolength{\evensidemargin}{-0.5in}
\addtolength{\textwidth}{1in}
\addtolength{\topmargin}{-.5in}
\addtolength{\textheight}{1.0in}

\urlstyle{same}

\raggedbottom
\raggedright
\setlength{\tabcolsep}{0in}

% Sections formatting
\titleformat{\section}{
  \vspace{-4pt}\scshape\raggedright\large
}{}{0em}{}[\color{black}\titlerule \vspace{-5pt}]

% Ensure that generate pdf is machine readable/ATS parsable
% \pdfgentounicode=1

%-------------------------
% Custom commands
\newcommand{\resumeItem}[1]{
  \item\small{
    {#1 \vspace{-2pt}}
  }
}

\newcommand{\resumeSubheading}[4]{
  \vspace{-2pt}\item
    \begin{tabular*}{0.97\textwidth}[t]{l@{\extracolsep{\fill}}r}
      \textbf{#1} & #2 \\
      \textit{\small#3} & \textit{\small #4} \\
    \end{tabular*}\vspace{-7pt}
}

\newcommand{\resumeSubSubheading}[2]{
    \item
    \begin{tabular*}{0.97\textwidth}{l@{\extracolsep{\fill}}r}
      \textit{\small#1} & \textit{\small #2} \\
    \end{tabular*}\vspace{-7pt}
}

\newcommand{\resumeProjectHeading}[2]{
    \item
    \begin{tabular*}{0.97\textwidth}{l@{\extracolsep{\fill}}r}
      \textbf{#1} & #2 \\
    \end{tabular*}\vspace{-7pt}
}

\newcommand{\resumeSubItem}[1]{\resumeItem{#1}\vspace{-4pt}}

\renewcommand\labelitemii{$\vcenter{\hbox{\tiny$\bullet$}}$}

\newcommand{\resumeSubHeadingListStart}{\begin{itemize}[leftmargin=0.15in, label={}]}
\newcommand{\resumeSubHeadingListEnd}{\end{itemize}}
\newcommand{\resumeItemListStart}{\begin{itemize}}
\newcommand{\resumeItemListEnd}{\end{itemize}\vspace{-5pt}}

%-------------------------------------------
%%%%%%  RESUME STARTS HERE  %%%%%%%%%%%%%%%%%%%%%%%%%%%%


\begin{document}

\begin{center}
    \textbf{\Huge \scshape JiaWei Lee} \\ \vspace{3pt}
    \faMobile \hspace{.5pt} \href{tel:0912911672}{\color{blue}+886 912 911 672} $|$
    \faAt \hspace{.5pt} \href{mailto:qw960211@gmail.com}{\color{blue}qw960211@gmail.com} $|$
    \faGithub \hspace{.5pt} \href{https://github.com/EC404}{\color{blue}GitHub}
\end{center}

{I am Joe Lee, a software developer with 3 years of experience at Turing-Drive, dedicated to research and implement path planning, system integration, and GNSS positioning. My coding skills include C++, C, and Python. I have implemented the algorithms on buses and golf carts.} \\

%-----------PROGRAMMING SKILLS-----------
\section{\textbf{Skills}}
 \begin{itemize}[leftmargin=0.15in, label={}]
    \small{\item{
     \textbf{Languages}{: C/C++, Python, Shell script} \\
     \textbf{Frameworks}{: ROS, ROS2} \\
     \textbf{OS}{: Linux, Window} \\
     \textbf{Developer Tools}{: Git, Docker, VS Code, Vim, Cmake, } \\
     \textbf{Libraries}{: Eigen, Opencv, Pandas, NumPy, Matplotlib} \\
     \textbf{Hardware}{: NVIDIA Jetson, Raspberry Pi, Arduino}
    }}
 \end{itemize}

%-----------EDUCATION-----------
\section{\textbf{Education}}
  \vspace{3pt}
  \resumeSubHeadingListStart
    \resumeSubheading
      {National Yunlin University of  Science and Technology}{Yunlin, Taiwan}
      {Department of Electrical Engineering}{Sep. 2017 -- Jun 2019}
      \resumeItemListStart
        \resumeItem{LiDAR Pedestrian Detection and Trajectory Tracking Based on Morphological Extended-Jump-Distance  Clustering Segmentation and Its Intelligent Patrolling  Security Robot Application}
      \resumeItemListEnd
    \resumeSubheading
      {National Formosa University}{Yunlin, Taiwan}
      {Department of Electrical Engineering}{Sep. 2015 -- Jun 2017}
  \resumeSubHeadingListEnd


%-----------EXPERIENCE-----------
\section{\textbf{Professional Experience}}
  \vspace{3pt}
  \resumeSubHeadingListStart

    \resumeSubheading 
      {\href{https://turing-drive.com/zh/home-zh/}{\color{blue}Turing Drive Inc}\emph{\scriptsize{(C++/C, Python, ROS, Shell, Linux, Path Planning, GNSS)}}}{Taipei, Taiwan}
      {Software Engineer}{Nov 2019 -- Present}
      \resumeItemListStart
      \resumeItem{\underline{\textbf{Open Planner Module:}} We 6-meter autonomous driving bus is put to service in  residential area and bus lane, on open road with mixed traffic flow. We integrated calibrated status of traffic light and perception module into the autonomous driving system. Total mileage exceeds 5,000 km, and over 4,000 people have been onboard.}
      \resumeItemListStart
        \resumeItem{Designed and implemented a logger system for the open planner, which records trajectory, controls information and creates visualizations. This reduced on-site debugging and parameter tuning time by 50\%.}
        \resumeItem{Comprehend open local planner, including trajectory generator, evluator, predictor and parameters, Since the functions used by generator and evaluation are too repetitive, they have been refactored and optimized. This resulted in a 60\% increase in execution speed.}
        \resumeItem{Participated in designing and implementing the Automatic Emergency Braking(AEB) and Adaptive Cruise Control(ACC) functions, which generates safe braking distances based on time-to-collision(TTC).}
      \resumeItemListEnd
      \resumeItem{\underline{\textbf{Motion Planning Algorithms:}} Comprehend motion planning algorithms and tuned parameters, including trajectory smoothing, generation of kinematically feasible and collision-free trajectories, vehicle cruising, and obstacle avoidance.}
      \resumeItem{\underline{\textbf{Motion Velocity Optimization Algorithms:}} Participated in designing and implementing a motion velocity optimization algorithm with setting objective function and constraints to generate velocity profile.}
      \resumeItem{\underline{\textbf{Upgrade and System Integration:}} Upgraded the autonomous driving system from ROS to ROS2 and integrated different information from various modules into system management, including state machine, error monitor and system interface. Comprehend the workflow of the autonomous driving system.}
      \resumeItem{\underline{\textbf{GNSS:}} Set up RTK infrastructure, designed and implemented GNSS communication parser, currently operating in various experimental sites.}
      % \resumeItem{Implemented leg detection and integrated it into the autonomous driving system for golf carts}
      % \resumeItem{Implemented the fusion of camera and 3D LiDAR sensors to provide actual depth information that is not detected by the camera}
    \resumeItemListEnd
  \resumeSubHeadingListEnd
  
%-----------PROJECTS-----------
\section{\textbf{Selected Projects \& Awards}}
  \vspace{3pt}
    \resumeSubHeadingListStart

      \resumeProjectHeading
        {\href{https://drive.google.com/file/d/1FCTHbe_6uXDAoiZOx13j0Z9KNvcpPl73/view?usp=sharing}{\color{blue}Smart trash can}\emph{\scriptsize{(C++/C, ROS, Navigation Stack, SLAM, Opencv, PCL)}}}{Sep 2017 -- May 2019}
          \resumeItemListStart
            \resumeItem{\underline{\textbf{Award:}} Applied to autonomous patrol security robots. It won a Silver Medal at the Korean Invention Awards and also achieved good results in domestic software competitions}
            \resumeItem{\underline{\textbf{System Integration:}} Integrated the SLAM module, Navigation Stack module, and leg tracking module into the robot system using ROS.}
            \resumeItem{\underline{\textbf{Leg Tracking:}} Researched and implemented point cloud clustering and geometric features of point clouds, Used Adaboost to implement leg detection and achieved pedestrian tracking through Kalman filtering. The detection performance is excellent in open spaces.}
          \resumeItemListEnd
    \resumeSubHeadingListEnd

%

%-------------------------------------------
\end{document}